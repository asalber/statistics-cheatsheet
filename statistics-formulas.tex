% !TEX root = statistics-formulas-cheatsheet.tex
% Author: Alfredo Sánchez Alberca (asalber@ceu.es)

\sloppy

\section*{Statistics Formulas}

\footnotesize
\tcbset{enhanced, colback=color1!10, colframe=color1, fonttitle=\bfseries\large\sffamily}

\begin{multicols*}{2}

\subsection*{Descriptive Statistics}

\begin{tcolorbox}[hbox, title=Frequencies]
\begin{minipage}{0.4\textwidth}
\begin{description}
\item [Sample size] $n$ num of individuals in the sample.
\end{description}
\begin{description}
\item [Absolute frequency] $n_i$ (num of $x_i$ in the sample)
\item [Relative frequency] $f_i=n_i/n$
\item [Cumulative absolute freq] $N_i=\sum_{k=0}^in_i$
\item [Cumulative relative freq] $F_i=N_i/n$
\end{description}
\end{minipage}
\end{tcolorbox}

\begin{tcolorbox}[hbox, title=Central tendency statistics]
\begin{minipage}{0.4\textwidth}
\begin{description}
\item [Mean] $\bar{x}=\dfrac{\sum x_in_i}{n}$
\item [Median] $me$ The value with cum.rel.freq. $F_{me}=0.5$.
\item [Mode] $mo$ The most frequent value.
\end{description}
\end{minipage}
\end{tcolorbox}

\begin{tcolorbox}[hbox, title=Position statistics]
\begin{minipage}{0.4\textwidth}
\begin{description}
\item [Quartiles] $Q_1,Q_2,Q_3$ divide the distribution into 4 equal parts.
      Their cum.rel.freqs. are
      $F_{Q_1}=0.25$, $F_{Q_2}=0.5$ and $F_{Q_3}=0.75$.
\item [Percentiles] $P_1,P_2,\cdots,P_{99}$ divide the distribution into 100 equal parts.\\
      The cum.rel.freq. is $F_{P_i}=i/100$.
\end{description}
\end{minipage}
\end{tcolorbox}

\begin{tcolorbox}[hbox, title=Dispersion statistics]
\begin{minipage}{0.4\textwidth}
\begin{description}
\item [Interquartile range] $IQR=Q_3-Q_1$
\item [Variance] $s^2=\dfrac{\sum (x_i-\bar x)^2n_i}{n}=\dfrac{\sum x_i^2n_i}{n}-\bar x^2$
\item [Standard deviation] $s=+\sqrt{s^2}$
\item [Coefficient of variation] $cv=\dfrac{s}{|\bar{x}|}$
\end{description}
\end{minipage}
\end{tcolorbox}

\begin{tcolorbox}[hbox, title=Shape statistics]
\begin{minipage}{0.4\textwidth}
\begin{description}
\item [Coefficient of skewness] $g_1=\dfrac{\sum(x_i-\bar{x})^3f_i}{s^3}$
\item [Coefficient of kurtosis] $g_2=\dfrac{\sum(x_i-\bar{x})^4f_i}{s^4}-3$
\end{description}
\end{minipage}
\end{tcolorbox}

\begin{tcolorbox}[hbox, title=Linear transformations]
\begin{minipage}{0.4\textwidth}
\begin{description}
\item[Linear transformation] $y=a+bx$
      \begin{align*}
      \bar y & = a+b\bar x \\
      s_y    & = bs_x
      \end{align*}
\item[Standarization] $z=\dfrac{x-\bar x}{s_x}$
\end{description}
\end{minipage}
\end{tcolorbox}


\subsection*{Regression and correlation}

\begin{tcolorbox}[hbox, title=Linear regression]
\begin{minipage}{0.4\textwidth}
\begin{description}
\item [Covariance] $s_{xy}=\dfrac{\sum x_iy_jn_{ij}}{n}-\bar{x}\bar{y}$
\item [Regression lines]:
      \begin{align*}
      \mbox{$y$ on $x$} & : y=\bar{y}+\dfrac{s_{xy}}{s_x^2}(x-\bar{x}) \\
      \mbox{$x$ on $y$} & : x=\bar{x}+\dfrac{s_{xy}}{s_y^2}(y-\bar{y})
      \end{align*}
\item [Regression coefficients]
      \[
      \mbox{($y$ on $x$) } b_{yx}=\dfrac{s_{xy}}{s_x^2}\quad \mbox{($x$ on
      $y$) } b_{xy}=\dfrac{s_{xy}}{s_y^2}
      \]
\item[Coefficient of determination]
      \[r^2=\dfrac{s_{xy}^2}{s_x^2s_y^2} \qquad 0\leq r^2\leq 1\]
\item[Correlation coefficient]
      \[r=\dfrac{s_{xy}}{s_xs_y}.\qquad -1\leq r\leq 1\]
\end{description}
\end{minipage}
\end{tcolorbox}

\medskip

\begin{tcolorbox}[hbox, title=Non-linear regression]
\begin{minipage}{0.4\textwidth}
\begin{description}
\item[Exponential model] $y=e^{a+bx}$\\
      Apply the logarithm to the dependent variable and compute the line $\log y = a+bx$.
\item[Logarithmic model] $y=a+b\log x$\\
      Apply the logarithm to the independent variable and compute the line $y=a+b\log x$.
\item[Potential model] $y=ax^b$\\
      Apply the logarithm to both variables and compute the line $\log y = a+b\log x$.
\end{description}
\end{minipage}
\end{tcolorbox}

\newpage

\subsection*{Probability}

\begin{tcolorbox}[hbox, title=Event operations]
\begin{minipage}{0.4\textwidth}
\textbf{Union}
\begin{center}
% Author: Alfredo Sánchez Alberca (asalber@ceu.es)
\begin{tikzpicture}
\def\firstcircle{(1.5,1.5) circle (1cm)}
\def\secondcircle{(2.5,1.5) circle (1cm)}

\fill[color1!30] \firstcircle;
\fill[color1!30] \secondcircle;
\draw (0,3) node[anchor=north east] {$\Omega$} rectangle (4,0);
\draw \firstcircle node[xshift=-0.9cm, yshift=0.9cm] {$A$};
\draw \secondcircle node[xshift=0.9cm, yshift=0.9cm] {$B$};

\node at (2,0.3) {$A\cup B$};
\end{tikzpicture}
\end{center}
\textbf{Intersection}
\begin{center}
% Author: Alfredo Sánchez Alberca (asalber@ceu.es)

\begin{tikzpicture}
\def\firstcircle{(1.5,1.5) circle (1cm)}
\def\secondcircle{(2.5,1.5) circle (1cm)}

\begin{scope}
\clip \firstcircle;
\fill[color1!30] \secondcircle;
\end{scope}

\draw (0,3) node[anchor=north east] {$\Omega$} rectangle (4,0);
\draw \firstcircle node[xshift=-0.9cm, yshift=0.9cm] {$A$};
\draw \secondcircle node[xshift=0.9cm, yshift=0.9cm] {$B$};

\node at (2,1.5) {$A\cap B$};
\end{tikzpicture}
\end{center}
\textbf{Complement}
\begin{center}
% Author: Alfredo Sánchez Alberca (asalber@ceu.es)

\begin{tikzpicture}
\def\circle{(1.5,1.5) circle (1cm)}
\def\rectangle{(4,0) rectangle (0,3)}

\begin{scope}[even odd rule]
\clip \circle (0,0) rectangle (4,3);
\fill[color1!30] \rectangle;
\end{scope}

\draw \rectangle node[anchor=north east] {$\Omega$};
\draw \circle node {$A$};
\node at (3,1.5) {$\overline A$};
\end{tikzpicture}
\end{center}
\textbf{Difference}
\begin{center}
% Author: Alfredo Sánchez Alberca (asalber@ceu.es)

\begin{tikzpicture}
\def\firstcircle{(1.5,1.5) circle (1cm)}
\def\secondcircle{(2.5,1.5) circle (1cm)}

\begin{scope}[even odd rule]
\clip \secondcircle (0,0) rectangle (4,3);
\fill[color1!30] \firstcircle;
\end{scope}

\draw (0,3) node[anchor=north east] {$\Omega$} rectangle (4,0);
\draw \firstcircle node[xshift=-0.9cm, yshift=0.9cm] {$A$};
\draw \secondcircle node[xshift=0.9cm, yshift=0.9cm] {$B$};

\node[anchor=east] at (1.5,1.5) {$A-B$};
\end{tikzpicture}
\end{center}
\end{minipage}
\end{tcolorbox}

\begin{tcolorbox}[hbox, title=Algebra of events]
\begin{minipage}{0.4\textwidth}
\begin{description}
\item[Idempotency] $A\cup A=A$,\quad $A\cap A=A$
\item[Commutative] $A\cup B=B\cup A$,\quad $A\cap B = B\cap A$
\item[Associative] $(A\cup B)\cup C = A\cup (B\cup C)$,\quad $(A\cap B)\cap C = A\cap (B\cap C)$
\item[Distributive] $(A\cup B)\cap C = (A\cap C)\cup (B\cap C)$,\quad $(A\cap B)\cup C = (A\cup C)\cap (B\cup C)$
\item[Neutral element] $A\cup \emptyset=A$,\quad $A\cap \Omega=A$
\item[Absorbing element] $A\cup \Omega=\Omega$,\quad $A\cap \emptyset=\emptyset$.
\item[Complementary symmetric element] $A\cup \overline A = \Omega$,\quad $A\cap \overline A= \emptyset$
\item[Double contrary] $\overline{\overline A} = A$
\item[Morgan's laws] $\overline{A\cup B} = \overline A\cap \overline B$,\quad $\overline{A\cap B} = \overline A\cup \overline B$
\end{description}
\end{minipage}
\end{tcolorbox}

\begin{tcolorbox}[hbox, title=Basic probability]
\begin{minipage}{0.4\textwidth}
\begin{description}
\item [Union] $P(A\cup B)=P(A)+P(B)-P(A\cap B)$
\item [Intersection] $P(A\cap B)=P(A)P(B|A)$
\item [Difference] $P(A-B)=P(A)-P(A\cap B)$
\item [Contrary] $P(\overline{A})=1-P(A)$
\end{description}
\end{minipage}
\end{tcolorbox}

\begin{tcolorbox}[hbox, title=Conditional probability]
\begin{minipage}{0.4\textwidth}
\begin{description}
\item [Conditional probability] $P(A|B)=\dfrac{P(A\cap B)}{P(B)}$
\item [Independent events] $P(A|B)=P(A)$.
\item [Total probability Theorem] \[P(B)=\sum_{i=1}^n P(A_i)P(B|A_i)\]
\item [Bayes Theorem] \[P(A_i|B)=\dfrac{P(A_i)P(B|A_i)}{\sum_{i=1}^n P(A_i)P(B|A_i)}\]
\end{description}
\end{minipage}
\end{tcolorbox}


\begin{tcolorbox}[hbox, title=Risks]
\begin{minipage}{0.4\textwidth}
\begin{center}
\begin{tabular}{|l|c|c|}
\cline{2-3}
\multicolumn{1}{c|}{} & $E$ & $\overline E$ \\
\hline
Treatment             & $a$ & $b$           \\
\hline
Control               & $c$ & $d$           \\
\hline
\end{tabular}
\end{center}
\begin{description}
\item[Prevalence] Proportion of individuals with $E$: $P(E)$
\item[Incidence rate or absolute risk] $R(E)=\dfrac{a}{a+b}$
\item[Odds] $O(E)=\dfrac{a}{b}$
\item[Relative risk] $RR(E)=\dfrac{a/(a+b)}{c/(c+d)}$
\item[Odds ratio] $OR(E)=\dfrac{a/b}{c/d}=\dfrac{a\cdot d}{b\cdot c}$
\end{description}
\end{minipage}
\end{tcolorbox}


\begin{tcolorbox}[hbox, title=Diagnostic tests]
\begin{minipage}{0.4\textwidth}
\begin{center}
\begin{tabular}{|l|c|c|}
\cline{2-3}
\multicolumn{1}{c|}{} & Disease $D$ & No disease $\overline D$ \\
\hline
Test $+$              & $VP$        & $FP$                     \\
\hline
Test $-$              & $FN$        & $VN$                     \\
\hline
\end{tabular}
\end{center}
\begin{description}
\item[Sensitivity] $P(+|D)=\dfrac{VP}{VP+FN}$
\item[Specificity] $P(-|\overline{D})=\dfrac{VN}{FP+VN}$
\item[Positive Predictive Value (PPV)] $P(D|+)=\dfrac{VP}{VP+FP}$
\item[Negative Predictive Value (NPV)] $P(\overline{D}|-)=\dfrac{VN}{FN+VN}$
\item[Positive Likelihood Ratio (LR+)] $\dfrac{P(+|D)}{P(+|\overline{D})}$
\item[Negative Likelihood Ratio (LR-)] $\dfrac{P(-|D)}{P(-|\overline{D})}$
\end{description}
\end{minipage}
\end{tcolorbox}


\subsection*{Random Variables}

\begin{tcolorbox}[hbox, title=Discrete]
\begin{minipage}{0.4\textwidth}
\begin{description}
\item [Binomial probability function $B(n,p)$]
      \[f(x)=\binom{n}{x}p^x (1-p)^{n-x}=\dfrac{n!}{x!(n-x)!}p^x (1-p)^{n-x}\]
\item [Poisson probability function $P(\lambda)$]
      \[f(x)=e^{-\lambda}\frac{\lambda^x}{x!}\]
\item [Law of rare events] $B(n,p)\approx P(np)$ for $n\geq 30$ and $p\leq 0.1$.
\end{description}
\end{minipage}
\end{tcolorbox}

\begin{tcolorbox}[hbox, title=Continuous]
\begin{minipage}{0.4\textwidth}
\begin{description}
\item[Normal $N(\mu,\sigma)$]
      \[f(x)= \frac{1}{\sigma\sqrt{2\pi}}e^{-\frac{(x-\mu)^2}{2\sigma^2}}\]
      \textbf{Standard normal $N(0,1)$}
\item[Chi-square $\chi^2(n)$]
      \[X = Z_1^2+\cdots +Z_n^2,\]
      where $Z_i\sim N(0,1)$.
\item[Student's t $T(n)$]
      \[T = \frac{Z}{\sqrt{X/n}},\]
      where $Z\sim N(0,1)$ and $X\sim \chi^2(n)$.
\item[Fisher's F $F(n,m)$]
      \[F = \frac{X/m}{Y/n},\]
      where $X\sim \chi^2(m)$ and $Y\sim \chi^2(n)$.
\end{description}
\end{minipage}
\end{tcolorbox}

\end{multicols*}